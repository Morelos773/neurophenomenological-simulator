\documentclass[11pt,a4paper]{article}
\usepackage[utf8]{inputenc}
\usepackage[spanish,english]{babel}
\usepackage{geometry}
\usepackage{hyperref}
\usepackage{xcolor}
\usepackage{listings}
\usepackage{graphicx}
\usepackage{fancyhdr}
\usepackage{titlesec}
\usepackage{booktabs}
\usepackage{amsmath}

% Fix headheight warning
\setlength{\headheight}{14pt}

% Page geometry
\geometry{margin=2.5cm}

% Colors
\definecolor{blue}{RGB}{0,82,155}
\definecolor{gray}{RGB}{100,100,100}

% Header and footer
\pagestyle{fancy}
\fancyhf{}
\fancyhead[L]{\textcolor{gray}{Neurophenomenological Simulator v2.0}}
\fancyhead[R]{\textcolor{gray}{Software Overview}}
\fancyfoot[C]{\textcolor{gray}{\thepage}}

% Title formatting
\titleformat{\section}{\Large\bfseries\color{blue}}{}{0em}{}
\titleformat{\subsection}{\large\bfseries\color{blue}}{}{0em}{}
\titleformat{\subsubsection}{\bfseries\color{blue}}{}{0em}{}

% Code listing style
\lstset{
	basicstyle=\ttfamily\small,
	breaklines=true,
	frame=single,
	numbers=left,
	numberstyle=\tiny\color{gray}
}

\begin{document}
	
	% Title page
	\begin{titlepage}
		\centering
		
		\vspace*{2cm}
		{\Huge\bfseries\color{blue} Neurophenomenological Simulator v2.0\par}
		\vspace{0.5cm}
		{\LARGE Advanced Phenomenological Coefficients Theory\par}
		
		\vspace{2cm}
		% Remove image or add a simple graphic
		\textcolor{blue}{\rule{0.3\textwidth}{2pt}}
		\vspace{0.5cm}
	%	{\large [Software Architecture Diagram]}\par
		
		\vspace{2cm}
		{\Large Marco Antonio Morelos Navidad\par}
		\vspace{0.2cm}
		{\large ORCID: 0009-0007-0083-5496\par}
		\vspace{0.2cm}
		{\large Estudiante Licenciatura Psicología Biomédica\par}
		\vspace{0.2cm}
		{\large Universidad Autónoma Metropolitana\par}
		
		\vspace{1cm}
		{\large \textbf{Release:} December 2024\par}
		{\large \textbf{Version:} 2.0\par}
		{\large \textbf{License:} MIT\par}
		
		\vfill
		{\large DOI: \url{https://doi.org/10.5281/zenodo.17689427}\par}
	\end{titlepage}
	
	% Clear header for TOC
	\thispagestyle{empty}
	\tableofcontents
	\newpage
	
	% Restore header
	\pagestyle{fancy}
	
	\section{Executive Summary}
	
	The \textbf{Neurophenomenological Simulator v2.0} represents a major advancement in computational consciousness research, implementing a complete mathematical framework with rigorous scientific validation. This release introduces advanced validation mechanisms, mind-brain isomorphism analysis, and comprehensive experimental testing capabilities.
	
	\section{Key Innovations in v2.0}
	
	\subsection{Advanced Scientific Validation}
	\begin{itemize}
		\item \textbf{Isomorphism Analysis}: Implementation of Theorem 3 (Mind-Brain Isomorphism $\Phi: \mathcal{H} \rightarrow \mathcal{N}$)
		\item \textbf{Statistical Validation}: Pearson correlation, ROC analysis, hysteresis detection
		\item \textbf{Automated Reporting}: Comprehensive scientific validation reports
	\end{itemize}
	
	\subsection{Enhanced Computational Framework}
	\begin{itemize}
		\item \textbf{Dual Simulation Modes}: Basic research \& advanced validation
		\item \textbf{Containerized Execution}: Docker support for full reproducibility
		\item \textbf{Robust Error Handling}: Professional-grade implementation
	\end{itemize}
	
	\subsection{Complete Documentation}
	\begin{itemize}
		\item Theoretical framework specification
		\item API references and usage examples
		\item Validation methodology documentation
		\item Installation and troubleshooting guides
	\end{itemize}
	
	\section{Technical Specifications}
	
	\begin{table}[h]
		\centering
		\begin{tabular}{p{0.3\textwidth} p{0.6\textwidth}}
			\toprule
			\textbf{Category} & \textbf{Specifications} \\
			\midrule
			Programming Language & Python 3.8+ \\
			Dependencies & NumPy, SciPy, Streamlit, Plotly, scikit-learn \\
			Architecture & Modular, extensible design \\
			Validation Methods & Statistical testing, ROC analysis, correlation \\
			Reproducibility & Docker containers, version control, automated tests \\
			License & MIT Open Source \\
			\bottomrule
		\end{tabular}
		\caption{Technical specifications of the simulator}
	\end{table}
	
	\section{Quick Start Guide}
	
	\subsection{Basic Installation}
	\begin{lstlisting}[language=bash]
		# Install required dependencies
		pip install numpy matplotlib streamlit scipy plotly pandas scikit-learn
		
		# Run the simulator
		streamlit run main_actual.py
	\end{lstlisting}
	
	\subsection{Access the Interface}
	\begin{itemize}
		\item Open web browser to: \url{http://localhost:8501}
		\item Configure parameters in sidebar
		\item Select experimental paradigm
		\item Execute simulation
	\end{itemize}
	
	\section{Experimental Paradigms}
	
	The simulator includes four experimental paradigms:
	
	\begin{enumerate}
		\item \textbf{validacion\_gamma}: Gamma synchronization validation
		\item \textbf{anestesia\_general}: General anesthesia simulation
		\item \textbf{microestimulacion}: Microstimulation effects analysis
		\item \textbf{basal}: Baseline conscious state
	\end{enumerate}
	
	\section{Scientific Validation Framework}
	
	\subsection{Validation Metrics}
	\begin{table}[h]
		\centering
		\begin{tabular}{p{0.4\textwidth} p{0.3\textwidth} p{0.2\textwidth}}
			\toprule
			\textbf{Validation Type} & \textbf{Expected Value} & \textbf{Threshold} \\
			\midrule
			$\Gamma$-|c| correlation & $r > 0.7$ & Statistical \\
			Threshold accuracy & $>85\%$ & Empirical \\
			AUC in ROC analysis & $>0.9$ & Statistical \\
			Hysteresis detection & Present & Qualitative \\
			\bottomrule
		\end{tabular}
		\caption{Expected validation results}
	\end{table}
	
	\subsection{Statistical Methods}
	\begin{itemize}
		\item \textbf{Pearson Correlation}: $\Gamma(t)$ vs $|c(t)|$ isomorphism
		\item \textbf{ROC Analysis}: Phenomenological threshold validation
		\item \textbf{T-tests}: Conscious vs unconscious state differences
		\item \textbf{Regression Analysis}: Coefficient relationships
	\end{itemize}
	
	\section{Access and Citation}
	
	\subsection{Software Access}
	\begin{itemize}
		\item \textbf{Primary DOI}: \url{https://doi.org/10.5281/zenodo.17689427}
		\item \textbf{Theoretical Framework}: \url{https://doi.org/10.5281/zenodo.17596542}
		\item \textbf{All Versions}: \url{https://doi.org/10.5281/zenodo.17619906}
	\end{itemize}
	
	\subsection{Recommended Citation}
	\begin{lstlisting}[language=TeX]
		@software{morelos_navidad_2024_17689427,
			author       = {Marco Antonio Morelos Navidad},
			title        = {Neurophenomenological Simulator v2.0: Advanced 
				Phenomenological Coefficients Theory},
			year         = 2024,
			publisher    = {Zenodo},
			doi          = {10.5281/zenodo.17689427},
			url          = {https://doi.org/10.5281/zenodo.17689427}
		}
	\end{lstlisting}
	
	\section{Research Applications}
	
	\subsection{Primary Research Areas}
	\begin{itemize}
		\item Computational neuroscience research
		\item Consciousness studies and modeling
		\item Neural correlates of consciousness
		\item Theoretical neuroscience development
	\end{itemize}
	
	\subsection{Educational Uses}
	\begin{itemize}
		\item Teaching computational neuroscience
		\item Consciousness theory courses
		\item Research methodology training
		\item Open science practices
	\end{itemize}
	
	\section{System Requirements}
	
	\subsection{Minimum Requirements}
	\begin{itemize}
		\item Python 3.8 or higher
		\item 4GB RAM
		\item Modern web browser
		\item 500MB disk space
	\end{itemize}
	
	\subsection{Recommended Specifications}
	\begin{itemize}
		\item 8GB RAM for large simulations
		\item Multi-core processor
		\item Stable internet connection
		\item GPU acceleration (optional)
	\end{itemize}
	
	\section{Support and Documentation}
	
	\subsection{Available Resources}
	\begin{itemize}
		\item \textbf{Complete Documentation}: In-package comprehensive guides
		\item \textbf{Usage Examples}: Basic to advanced workflows
		\item \textbf{API Reference}: Detailed class and method documentation
		\item \textbf{Validation Methods}: Scientific validation protocols
	\end{itemize}
	
	\subsection{Community Support}
	\begin{itemize}
		\item ResearchGate project page
		\item GitHub repository issues
		\item Academic collaboration opportunities
		\item Email contact for technical support
	\end{itemize}
	
	\section{Theoretical Framework}
	
	\subsection{Mathematical Foundations}
	The simulator implements the complete Phenomenological Coefficients Theory:
	
	\begin{itemize}
		\item \textbf{Axiom 1}: Conscious state space $\mathcal{H}$ with orthonormal basis
		\item \textbf{Axiom 2}: Phenomenological coefficients $c_i(t) = \Gamma_i(t) \cdot A_i(t) \cdot e^{i\theta_i(t)}$
		\item \textbf{Theorem 1}: Unified conscious field $|\Psi(t)\rangle = \sum c_i(t)|\psi_i\rangle$
		\item \textbf{Theorem 2}: Resource conservation $\langle\Psi|\Psi\rangle \leq C_{\text{max}}$
		\item \textbf{Theorem 3}: Mind-brain isomorphism $\Phi: \mathcal{H} \rightarrow \mathcal{N}$
	\end{itemize}
	
	\section{License and Availability}
	
	\subsection{Open Source Commitment}
	\begin{itemize}
		\item \textbf{License}: MIT Open Source
		\item \textbf{Access}: No restrictions
		\item \textbf{Modification}: Allowed with attribution
		\item \textbf{Distribution}: Free redistribution
	\end{itemize}
	
	\subsection{Long-term Preservation}
	\begin{itemize}
		\item Permanent Zenodo archival
		\item Version-controlled releases
		\item DOI persistence
		\item Community maintenance
	\end{itemize}
	
	\begin{center}
		\rule{0.8\textwidth}{0.5pt}
		
		\textit{This software represents a significant advancement in computational consciousness research with rigorous scientific validation methodologies.}
		
		\vspace{0.5cm}
		\textbf{Contact:} Marco Antonio Morelos Navidad \\
		\textbf{ORCID:} 0009-0007-0083-5496 \\
		\textbf{Email:} 2173084147@correo.ler.uam.mx
	\end{center}
	
\end{document}